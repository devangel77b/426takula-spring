\documentclass[12pt,conference,onecolumn]{IEEEtran}

\usepackage[hidelinks]{hyperref}

\title{Visualizing sidereal and solar time in honor of Karl Jansky: Internship at IEEE, AT\&T, and Bell Labs}
\author{%
\IEEEauthorblockN{Cole Canada}\IEEEauthorblockA{Science \& Engineering\\Manalapan High School\\Englishtown, NJ\\\href{mailto:426ccanada@frhsd.com}{426ccanada@frhsd.com}}\and
\IEEEauthorblockN{Tushaar Akula}\IEEEauthorblockA{Science \& Engineering\\Manalapan High School\\Englishtown, NJ\\\href{mailto:426takula@frhsd.com}{426takula@frhsd.com}}\and
\IEEEauthorblockN{K August}\IEEEauthorblockA{\href{mailto:kit.august@gmail.com}{kit.august@gmail.com}}\and
\IEEEauthorblockN{T M Willis III}\IEEEauthorblockA{\href{mailto:tw1431@att.com}{tw1431@att.com}}
}

\date{June 16, 2026}

\newcommand{\keywords}{AT\&T, Bell Labs, Karl Jansky, sidereal time, internship}

\usepackage{hyperref}
\makeatletter
\AtBeginDocument{
\hypersetup{%
pdftitle={\@title},
pdfauthor={Cole Canada, Tusheisty Akula, K August, and T M Willis III},
pdfkeywords={\keywords}}}
\makeatother

\begin{document}
\maketitle 

\begin{abstract}
To understand the significance of any discovery, one must grasp the historical context and collective efforts behind its emergence. Bell Labs, an institution famed for foundational advances in communications and physics, provided the environment where Karl Jansky first identified radio emissions from the Milky Way. This project supported by mentorship from the IEEE, AT\&T Labs, and Bell Labs employs interactive visualizations of Earth and Mars time, a street clock representing Martian sidereal and solar time, and a novel digital sundial to bring these discoveries to a broad public audience. The digital sundial, the project’s current focus, is designed to provide accurate digital timekeeping down to the minute. Furthermore, physical models are being developed for dual-sided street clocks destined for placement at the Dr. Robert Woodrow Wilson Park. These clocks, featuring vintage accessories, will allow viewers to directly compare solar and sidereal timekeeping systems. To engage younger audiences, we will utilize the Nano Banana AI model to help children visualize themselves on Mars, using technology to spark a lasting interest in astronomy. Our goal is to implement these prototypes into physical models that inspire the future by showcasing astronomical phenomena both old and new by the end of the semester. Ultimately, this project connects modern science to the historical roots of radio astronomy in a way that remains engaging and accessible to all ages.

%In order to understand the significance of any discovery, it is essential to understand the historical context in which it emerged and the shared effort that influenced its development. Bell Labs is an industrial research institution known for foundational advances in communications, physics, and engineering, which provided the environment in which Jansky first identified radio emissions from the Milky Way. This project through the mentorship with IEEE, AT\&T Labs, and Bell Labs employs interactive visualizations of time on Mars and Earth on a website, a street clock representation of Mars' sidereal and solar time, and an a novel digital sundial to bring his discoveries to life for a broad public audience. The website uses HTML, JavaScript, and p5.js to model planetary rotation and orbital motion while telling the full story of Karl Jansky’s life, discovery process, and contributions to radio astronomy. Designed for installation at the Dr. Robert Woodrow Wilson Park, the Bell Labs Horn Antenna site in Holmdel, the project features a dual-sided clock displaying Mars sidereal time on one side and Mars solar time on the other, allowing viewers to directly compare these two timekeeping systems. In addition, there will be an improved digital sundial, intended to become one of the largest and most accurate of its kind. Overall, the project connects modern science to the historical roots of radio astronomy in a way that is engaging and accessible to all ages. 
%In order to understand the significance of any discovery, it is essential to understand the historical context in which it emerged and the shared effort that influenced its development. Bell Labs is an industrial research institution known for foundational advances in communications, physics, and engineering, which provided the environment in which Jansky first identified radio emissions from the Milky Way. This mentorship with Bell Labs employs interactive visualizations of time on Mars and Earth on a website, a street clock representation of Mars' sidereal and solar time, and an upscaled digital sundial to bring his discoveries to life for a broad public audience. The website uses HTML, JavaScript, and p5.js to model planetary rotation and orbital motion while telling the full story of Karl Jansky’s life, discovery process, and contributions to radio astronomy. Designed for installation at the Bell Labs Horn Antenna site in Holmdel, the project features a dual-sided clock displaying Mars sidereal time on one side and Mars solar time on the other, allowing viewers to directly compare these two timekeeping systems. In addition, there will be an upscaled digital sundial, intended to become one of the largest and most accurate of its kind. Overall, the project connects modern science to the historical roots of radio astronomy in a way that is engaging and accessible to all ages. 
\end{abstract}

\begin{IEEEkeywords}
\keywords
\end{IEEEkeywords}

\end{document}
